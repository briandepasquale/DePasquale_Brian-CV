\documentclass[margin, 10pt]{res}

\usepackage{anysize}
\marginsize{1in}{0.75in}{1in}{0.5in}
\newsectionwidth{1in}

\usepackage{palatino}
%\usepackage[urw-garamond]{mathdesign}
%\usepackage[T1]{fontenc}
\usepackage{mathtools}
\usepackage{enumitem}
\usepackage{etaremune}
\usepackage[colorlinks=true,urlcolor=black]{hyperref}

\setlength{\textwidth}{5.25in}

\begin{document}

\hspace{-1.0in} {\LARGE\bf Brian DePasquale} \hfill\href{mailto:depasquale@princeton.edu}{depasquale@princeton.edu}
 
\moveleft\hoffset\vbox{\hrule width 6.25in height 1pt}\smallskip

\vspace{-3mm}
\hspace{-1.0in} Princeton Neuroscience Institute \hfill Tel: +1 914 474 0443\\

\vspace{-8mm}
\hspace{-1.0in} Washington Road \hfill Web: \href{http://www.columbia.edu/~bdd2107}{columbia.edu/$\sim$bdd2107} \\

\vspace{-8mm}
\hspace{-1.0in} Princeton, NJ 08544\\

\vspace{-3mm}
\moveleft\hoffset\vbox{\hrule width 6.25in height 1pt}

\begin{resume}

\vspace{-4mm}
\section{\textnormal\bf{Academic Positions}}

\textbf{Princeton University} \hfill Princeton, NJ\\
\vspace{-3mm}
\begin{itemize}[label={}]
\itemsep-0.25em 
\item Postdoctoral Research Associate \hfill 2016--present
\item Laboratories of Carlos Brody \& Jonathan Pillow\\
\end{itemize}

\vspace{-4mm}
\textbf{Columbia University} \hfill New York, NY\\
\vspace{-3mm}
\begin{itemize}[label={}]
\itemsep-0.25em 
\item Ph.D. in Neurobiology \& Behavior 
\hfill 2016
\item Laboratory of Larry Abbott\\
\end{itemize}

\vspace{-4mm}
\textbf{Massachusetts Institute of Technology} \hfill Cambridge, MA\\
\vspace{-3mm}
\begin{itemize}[label={}]
\itemsep-0.25em 
\item Research and Technical Assistant \hfill 2005--2009
\item Laboratory of Ann Graybiel\\
\end{itemize}

\vspace{-4mm}
\textbf{Fordham University} \hfill Bronx, NY\\
\vspace{-3mm}
\begin{itemize}[label={}]
\itemsep-0.25em 
\item B.S. in Physics, \emph{cum laude} \hfill 2005
\item Victor F. Hess Award (top graduating physics student)\\
\end{itemize}

\section{\textnormal\bf{Publications}} 

\begin{enumerate}[label={[\arabic*]}]
\itemsep0.5em 
\item DePasquale, B., Sussillo, D., Churchland, M.M., \& Abbott, L.F. (2018). Using recurrent neural networks to approximate low-dimensional, nonlinear population dynamics. \emph{in preparation}.
\item Panichello, M.F., DePasquale, B., Pillow, J.W. \& Buschman, T.J. (2018). Attractor dynamics mediate errors in visual working memory. \emph{submitted}.
\item Insanally, M.N., Carcea, I., Field, R.E., Rodgers, C., DePasquale, B., Rajan, K., DeWeese, M.R., Albanna, B.F. \& Froemke, R.C. (2018). Nominally non-responsive frontal and sensory cortical cells encode task-relevant variables via ensemble consensus-building. \emph{under review}.
\item DePasquale, B., Cueva, C.J., Rajan, K., Escola, G.S. \& Abbott, L.F. (2018). full-FORCE: A target-based method for training recurrent networks. \href{http://journals.plos.org/plosone/article?id=10.1371/journal.pone.0191527}{PLoS ONE 13(2): e0191527}.
\item DePasquale, B. (2016). \emph{Methods for Building Network Models of Neural Circuits.} \href{http://dx.doi.org/10.7916/D8W09600}{Ph.D. thesis, Columbia University}.
\item Abbott, L.F., DePasquale, B. \& Memmesheimer, R.-M. (2016). Building functional networks of spiking model neurons. \href{http://www.nature.com/neuro/journal/v19/n3/full/nn.4241.html}{Nature Neuroscience 19:350-355}.
\item DePasquale, B., Churchland, M.M. \& Abbott, L.F. (2016). Using firing-rate dynamics to train recurrent networks of spiking model neurons. \href{http://arxiv.org/abs/1601.07620} {arXiv:1601.07620}.
\item Feingold, J., Gibson, D.J., DePasquale, B. \& Graybiel, A.M. (2015). Bursts of beta oscillation differentiate postperformance activity in the striatum and motor cortex of monkeys performing movement tasks. \href{http://www.pnas.org/content/112/44/13687.long}{PNAS 112(44):13687-13692}.
\item Paninski, L., Vidne, M., DePasquale, B. \& Ferreira, D.G. (2012). Inferring synaptic inputs given a noisy voltage trace. \href{http://link.springer.com/article/10.1007\%2Fs10827-011-0371-7}{Journal of Computational Neuroscience 33:1-19}.
\end{enumerate}

\section{\textnormal\bf{Research Support}} 

\begin{itemize}[label = {}]
\itemsep-0.25em 
\item National Science Foundation Graduate Research Fellow \hfill 2010--2013
\end{itemize}

\newpage

\section{\textnormal\bf{Teaching \& Mentorship}} 

\textbf{Princeton University} \hfill Princeton, NJ\\
\vspace{-3mm}
\begin{itemize}[label={}]
\itemsep 0.75em 

\item Undergraduate research co-advisor \hfill 2016--present \\

\end{itemize}

\textbf{Columbia University} \hfill New York, NY\\
\vspace{-3mm}
\begin{itemize}[label={}]
\itemsep 0.75em 

\item Instructor, General Physics (undergraduate) \hfill 2013--2014 \\
\href{http://www.shpep.org}{Summer Health Professions Education Program}

\item TA, Introduction to Theoretical Neuroscience (graduate) \hfill 2011 \& 2013

\section{\textnormal\bf{Service}}

%\begin{itemize}[label = \tiny$\bullet$]
\begin{itemize}[label = {}]
\itemsep1em
\item Organizer, \emph{Recurrent Spiking Neural Networks---Dynamics, Learning, Computation}. COSYNE Workshop (2016), Salt Lake City, UT.
\item \emph{Ad hoc} reviewer: NIPS, Brain Research, eLife.
\end{itemize}

\section{\textnormal\bf{References}}

\begin{itemize}[label = {}]
\itemsep-0.5em 

\item Carlos Brody, Ph.D.\\
Wilbur H. Gantz III '59 Professor in Neuroscience\\
Princeton University\\
Tel: +1 609 258 7645, email: \href{mailto:brody@princeton.edu}{brody@princeton.edu}\\

\item Jonathan Pillow, Ph.D.\\
Associate Professor of Psychology\\
Princeton University\\
Tel: +1 609 258 7848, email: \href{mailto:pillow@princeton.edu}{pillow@princeton.edu}\\

\item Larry F. Abbott, Ph.D.\\
William Bloor Professor of Theoretical Neuroscience\\
Columbia University\\
Tel: +1 212 853 1065, email: \href{mailto:lfa2103@columbia.edu}{lfa2103@columbia.edu}\\

\item Mark M. Churchland, Ph.D.\\
Assistant Professor, Department of Neuroscience\\
Columbia University\\
Tel: +1 212 853 1068, email: \href{mailto:mc3502@columbia.edu}{mc3502@columbia.edu}\\

\item Ann M. Graybiel, Ph.D.\\
Institute Professor\\
Massachusetts Institute of Technology\\
Tel: +1 617 253 5785, email: \href{mailto:graybiel@mit.edu}{graybiel@mit.edu}\\

\end{itemize}

\end{itemize}

\newpage

\section{\textnormal\bf{Conference Proceedings \& Talks}} 

\begin{enumerate}[label={}]
\itemsep 0.75em 
\item Matthew Panichello, Brian DePasquale, Jonathan Pillow, Timothy Buschman (2018). Memory load modulates the dynamics of visual working memory. Vision Sciences Society 18$^{th}$ Annual Meeting, St. Pete Beach, FL.
\item Brian DePasquale, Mark M. Churchland, LF Abbott (2016). Using firing-rate dynamics to train recurrent spiking neural networks. Recurrent Spiking Neural Networks---Dynamics, Learning, Computation, COSYNE Workshop, Salt Lake City, UT.
\item Brian DePasquale, Christopher J. Cueva, Raoul-Martin Memmesheimer, LF Abbott, G. Sean Escola (2016). Full-rank regularized learning in recurrently connected firing rate networks. COSYNE, Salt Lake City, UT.
\item Brian DePasquale, Mark M. Churchland, LF Abbott (2015). Using firing-rate dynamics to train recurrent spiking neural networks. Annual Tri-Center Gatsby Meeting, Columbia University, NY, NY.
\item Brian DePasquale, Mark M. Churchland, LF Abbott (2014). Using firing rate dynamics to train spiking neural networks that perform tasks. Techniques and Approaches in Theoretical Neuroscience, Janelia Farms Research Campus, HHMI, Ashburn, VA.
\item Brian DePasquale, Mark M. Churchland, LF Abbott (2014). Constructing networks of spiking neurons that perform tasks. Department of Neuroscience retreat, Columbia University, NY, NY.
\item Brian DePasquale, Mark M. Churchland, LF Abbott (2014). Firing rate dynamics from spiking networks. COSYNE, Salt Lake City, UT.
\item Brian DePasquale, Mark M. Churchland, LF Abbott (2013). Low-rank connectivity induces firing rate fluctuations in a chaotic spiking model. Temporal Dynamics in Learning: Networks and Neural Data, Janelia Farms Research Campus, HHMI, Ashburn, VA.
\item Brian DePasquale, Mark M. Churchland, LF Abbott (2013). Low-rank connectivity induces firing rate fluctuations in a chaotic spiking model. COSYNE, Salt Lake City, UT.
\item J Feingold, Brian DePasquale, AM Graybiel (2009). Modulation of beta power in the prefrontal cortex and Caudate Nucleus of monkeys during self-timed sequential arm movements. SFN 39$^{th}$ Annual Meeting, Chicago, IL
\item J Feingold, Brian DePasquale, AM Graybiel (2007). Cortical 8-20 Hz oscillations in supplementary motor areas during self-timed sequential arm movements in monkey. SFN 37$^{th}$ Annual Meeting, San Diego, CA
\end{enumerate}

\end{resume}
\end{document}